\documentclass[12pt]{article}
\usepackage[utf8]{inputenc}
\usepackage{geometry}
\geometry{letterpaper, margin=0.25in}
\usepackage{graphicx} 
\usepackage{parskip}
\usepackage{booktabs}
\usepackage{array} 
\usepackage{paralist} 
\usepackage{verbatim}
\usepackage{subfig}
\usepackage{fancyhdr}
\usepackage{sectsty}
\usepackage[shortlabels]{enumitem}

\pagestyle{fancy}
\renewcommand{\headrulewidth}{0pt} 
\lhead{}\chead{}\rhead{}
\lfoot{}\cfoot{\thepage}\rfoot{}

%%% ToC (table of contents) APPEARANCE
\usepackage[nottoc,notlof,notlot]{tocbibind} 
\usepackage[titles,subfigure]{tocloft}
\renewcommand{\cftsecfont}{\rmfamily\mdseries\upshape}
\renewcommand{\cftsecpagefont}{\rmfamily\mdseries\upshape} %

\usepackage{amsmath}
\usepackage{amssymb}
\usepackage{mathtools}
\usepackage{empheq}
\usepackage{xcolor}
\usepackage{bbm}
\usepackage{tikz}
\usepackage{pgfplots}
\usepackage{tikz-cd}
\pgfplotsset{compat=1.18}

\newcommand{\ans}[1]{\boxed{\text{#1}}}
\newcommand{\vecs}[1]{\langle #1\rangle}
\renewcommand{\hat}[1]{\widehat{#1}}

\renewcommand{\P}{\mathbb{P}}
\newcommand{\R}{\mathbb{R}}
\newcommand{\E}{\mathbb{E}}
\newcommand{\Z}{\mathbb{Z}}
\newcommand{\N}{\mathbb{N}}
\newcommand{\Q}{\mathbb{Q}}
\newcommand{\C}{\mathbb{C}}

\newcommand{\ind}{\mathbbm{1}}
\newcommand{\qed}{\quad \blacksquare}

\newcommand{\brak}[1]{\left\langle #1 \right\rangle}
\newcommand{\bra}[1]{\left\langle #1 \right\vert}
\newcommand{\ket}[1]{\left\vert #1 \right\rangle}

\newcommand{\abs}[1]{\left\vert #1 \right\vert}
\newcommand{\mfX}{\mathfrak{X}}
\newcommand{\ep}{\varepsilon}

\newcommand{\Ec}{\mathcal{E}}
\newcommand{\A}{\mathcal{A}}
\newcommand{\F}{\mathcal{F}}
\newcommand{\Cc}{\mathcal{C}}
\newcommand{\B}{\mathcal{B}}
\newcommand{\M}{\mathcal{M}}
\newcommand{\X}{\chi}
\renewcommand{\L}{\mathcal{L}}

\newcommand{\sub}{\subseteq}
\newcommand{\st}{\text{ s.t. }}
\newcommand{\card}{\text{card }}
\renewcommand{\div}{\vspace*{10pt}\hrule\vspace*{10pt}}
\newcommand{\surj}{\twoheadrightarrow}
\newcommand{\inj}{\hookrightarrow}
\newcommand{\biject}{\hookrightarrow \hspace{-8pt} \rightarrow}
\renewcommand{\bar}[1]{\overline{#1}}
\newcommand{\overcirc}[1]{\overset{\circ}{#1}}
\newcommand{\diam}{\text{diam }}

\renewcommand{\Re}{\text{Re}\,}
\renewcommand{\Im}{\text{Im}\,}
\newcommand{\sign}{\text{sign}\,}

\newcommand*{\tbf}[1]{\ifmmode\mathbf{#1}\else\textbf{#1}\fi}

\usepackage{tcolorbox}
\tcbuselibrary{breakable, skins}
\tcbset{enhanced}
\newenvironment*{tbox}[2][gray]{
    \begin{tcolorbox}[
        parbox=false,
        colback=#1!5!white,
        colframe=#1!75!black,
        breakable,
        title={#2}
    ]}
    {\end{tcolorbox}}

\newenvironment*{exercise}[1][red]{
    \begin{tcolorbox}[
        parbox=false,
        colback=#1!5!white,
        colframe=#1!75!black,
        breakable
    ]}
    {\end{tcolorbox}}

\newenvironment*{proof}[1][blue]{
\begin{tcolorbox}[
    parbox=false,
    colback=#1!5!white,
    colframe=#1!75!black,
    breakable
]}
{\end{tcolorbox}}

\title{APMA 1360: Applied Dynamical Systems}
\author{Milan Capoor}
\date{Spring 2025}

\begin{document}
\maketitle

\section{Jan 22}
\subsection*{Motivations - Applications + Phenomena}
\begin{enumerate}
    \item \tbf{Bifurcation theory:} How do systems change as parameters change?

          \emph{Examples:}
          \begin{itemize}
              \item Mechanical systems (e.g. what will happen to a bead as an apparatus is rotated at velocity $\omega$?)
              \item Chemical reactions (e.g. Belusov-Zhabotinsky reaction - oscillations in chemical reactions)
              \item Tipping points (e.g. climate change, convection currents)
              \item Population dynamics (e.g. predator-prey models, outbreaks)
              \item Synchronization (e.g. firefly synchronous lighting, brain activity patterns)
              \item Chaotic dynamics (e.g. double pendulum)
          \end{itemize}

    \item \tbf{Existence and Uniqueness}

    \item \tbf{Dynamical theory}

    \item \tbf{Chaotic dynamics}
\end{enumerate}

\subsection*{Bifurcation Theory}
\tbf{Example (Overdamped bead on loop)}

\begin{center}
    \begin{tikzpicture}
        \draw (0,0) circle (2);
        \draw[fill, blue] ({-sqrt(2)},{-sqrt(2)}) circle (0.1);

        \draw (0, 3) -- (0, -3);
        \draw[->, dashed, red, rotate=-90] ([shift=(-150:0.5)]-2, 0) arc (-150:150:0.1 and 0.5);

    \end{tikzpicture}
\end{center}

\tbf{Goal:} What will happen to the bead as the loop is rotated at velocity $\omega$?

We assume that the only forces on the bead are gravitation, friction, and centrifugal force.

This gives a force diagram:
\begin{center}
    \begin{tikzpicture}

        \draw (0,0) circle (0.05);
        \draw[->] (0, 0) -- (0,-2) node[below] {$mg$};

        \draw (0, 0) arc[start angle =-45, end angle = 0, radius = 2];
        \draw (0, 0) arc[start angle =-45, end angle = -90, radius = 2];

        \draw[->, dashed] (-2,0) -- (0,0);
        \draw[->] (0, 0) -- (2, 0) node[right] {$mr\omega^2\sin \phi$};

        \draw[blue] (-2, -2) -- (2, 2) node[above, right] {Tangent to loop};

        \draw[green!60!black] (0.5, 0)  arc[start angle = 0, end angle = 45, radius = 0.5] node[right] {$\phi$};
    \end{tikzpicture}
\end{center}

From Newton's law,
\[\underbrace{mr \frac{d^2\phi}{dt^2}}_{\text{acceleration}} = -b \frac{d\phi}{dt} - mg \sin \phi + m\omega^2 r \sin\phi \cos \phi\]

Assuming $b \gg 1$, we can neglect the LHS so
\begin{align*}
    \frac{d\phi}{dt} & = -\frac{mg}{b}\sin \phi + \frac{m\omega^2 r}{b} \sin \phi \cos \phi    \\
                     & = \frac{mg}{b} \sin \phi \left(\frac{\omega^2 r}{g} \cos \phi- 1\right) \\
                     & = a \sin \phi (\mu \cos \phi - 1)
\end{align*}

\end{document}